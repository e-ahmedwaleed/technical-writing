%% Compile using pdflatex engine, TeX 2015 distro.
%% Template for ENG 401 reports
%% by Robin Turner
%% Adapted from the IEEE peer review template
%% https://www.overleaf.com/latex/templates/latex-template-for-technical-report/qtznkrpkjybm

\documentclass[peerreview]{IEEEtran}
\usepackage{cite}                    % Tidies up citation numbers.
\usepackage{url}                     % Provides better formatting of URLs.
\usepackage{booktabs}                % Allows the use of \toprule, \midrule and \bottomrule in tables for horizontal lines

\usepackage{graphicx}   % -> includes pngs
\usepackage{float}      % -> adds tables

% https://tex.stackexchange.com/questions/120291/arabic-in-latex
\usepackage{arabtex}
\usepackage{utf8}

% https://tex.stackexchange.com/questions/101964/change-font-size-in-arabtex-package
%%% from Uxnsh.fd
\DeclareFontFamily{U}{xnsh}{}%

\DeclareFontShape{U}{xnsh}{m}{n}{%                                   
   <-6> sfixed * [6.0] xnsh14
      <6-10> s * [1.20] xnsh14
         <10><10.95><12><14.4><17.28><20.74><24.88> s * [1.20] xnsh14
         }{}

\DeclareFontShape{U}{xnsh}{bx}{n}{%
   <-6> sfixed * [6.0] xnsh14bf
   <6-10> s * [1.20] xnsh14bf
   <10><10.95><12><14.4><17.28><20.74><24.88> s * [1.20] xnsh14bf
}{}
%%% end of added code

% https://tex.stackexchange.com/questions/3033/forcing-linebreaks-in-url
\PassOptionsToPackage{hyphens}{url}\usepackage{hyperref}

\hyphenation{op-tical net-works semi-conduc-tor} % Corrects some bad hyphenation 

\begin{document}

%start encoding to Unicode
%Note that your layout must support Arabic text when compiling
\setcode{utf8}
%To start typing in Arabic use the command arabtext

% \begin{titlepage}
% paper title
% can use line-breaks \\ within to get better formatting as desired
\title{An Easy Approach to Technical Writing}

% author names and affiliations
\author{Ahmed Waleed \\
Co-authors: Amr El Begawy, Marwan Ashraf \\
Department of Computer Engineering\\
Alexandria University\\
}
\date{December 29, 2019}

% make the title area
\maketitle
\tableofcontents
\listoffigures
\listoftables
% \end{titlepage}

\IEEEpeerreviewmaketitle

\begin{abstract}
% How to Write an Abstract for a Research Paper: https://youtu.be/JMEnRBss6V4
% Motivation
Technical reports are usually written according to general standards, corporate design standards of the current university or company, logical rules, and practical experiences.
% Problems
These rules are not known well enough among writing engineers and technicians. Therefore, this summary intends to help you create technical reports.
    Such an interesting topic, many students struggle to get started with.
% Methods
    Using rules and guidelines contained, you will be introduced to all the
    basic technical writing concepts.
\end{abstract}

\section{Technical Reports Writing\ \cite{Lec1}}
Technical writing is a style of formal writing used in many fields to explain technology and related ideas to:
\begin{itemize}
\item Specific technical, general technical or non-technical audiences.
\end{itemize}
\vspace{-5mm}   % Reduce white space

\subsection{Characteristics of Good Technical Writing}

% https://www.overleaf.com/learn/latex/Text_alignment
\begin{flushright}
    \leavevmode\\[-2mm]
    \begin{tabular}{|p{1.4cm} p{0.1cm} p{5.5cm}|}
        \hline
        \multicolumn{3}{c}{}\\[-1em]
        Accurate   &: & conforms to truth or fact.\\
        Clear      &: & simple and direct sentences.\\
        Concise    &: & use of a minimum of words.\\
        Coherent   &: & easy-to-follow line of thinking.\\
        Accessible &: & consists of small independent sections.\\
        \multicolumn{3}{c}{}\\[-1em]
        \hline
    \end{tabular}
\end{flushright}

\subsection{Converting to Technical Writing\ \cite{Style_Reference}}
Suppose you are writing a report where you may not use ``I'', or you are writing about a sentence subject that can not actually do anything. What to do when the passive voice is the best, most natural choice?
The answer lies in writing direct sentences—in passive voice—that have simple subjects and verbs.

% https://www.overleaf.com/learn/latex/Text_alignment
\begin{flushright}
    \small
    \leavevmode\\[-2mm]
    \begin{tabular}{|p{3.41cm} p{0.2cm} p{3.4cm}|}
        \hline
        \multicolumn{3}{c}{}\\[-3mm]
        I conducted a tensile test at room temperature because & $\rightarrow$ &  A tensile test was conducted.\\
        I needed to find out a baseline \ for \ tensile \ strength. & $\rightarrow$ &  Baseline tensile strength was established.\\
        I followed ASTM specification E8M for testing. & $\rightarrow$ &  ASTM Specification E8M was followed.\\
        \multicolumn{3}{c}{}\\[-3.5mm]
        \hline
    \end{tabular}
    \\[+1.5mm]
    Using passive voice does not have to create ambiguity.
\end{flushright}

\section{The Writing Process\ \cite{Lec2}}
Writing a message that is consistently well received can be hard for new writers to achieve. The three part writing process ensures the best outcome each time.
\begin{figure}[!h]
    \centering
    \includegraphics[width=0.8\columnwidth]{Writing-Process} 
    \caption{The Three-Part Writing Process\cite{lumen_course}}
\end{figure}

\subsection{Planning}
The most important part of the process requires a bit of time. This is also the most underused part of the process. When people do not plan thoughtfully, their writing becomes very disorganized. \cite{sethperler_video}

% https://www.overleaf.com/learn/latex/Lists
\begin{itemize}
   \item Analyzing Your Audience
   \begin{itemize}
     \item identify your readers, their attitudes, and expectations.
   \end{itemize}
   \item Analyzing Your Purpose
   \begin{itemize}
     \item what do you want them to know or to do? 
   \end{itemize}
   \item Generating Ideas about your Topic
   \begin{itemize}
     \item asking journalistic questions:
     \begin{itemize}
        \item who, what, when, where, why, and how.
     \end{itemize}
     \item write down your ideas:
     \begin{itemize}
        \item brainstorming or free writing.
     \end{itemize}
     \item talking them through:
     \begin{itemize}
        \item talking with someone.
     \end{itemize}
     \item imagine and visualize:
     \begin{itemize}
        \item clustering or branching.
     \end{itemize}
   \end{itemize}
   \item Researching Additional Information
   \begin{itemize}
   \item \small{(read)} \normalsize{references,} \small{(interview)} \normalsize{experts,} \small{(distribute)} \normalsize{surveys, or} \small{(conduct)} \normalsize{experiments.}
   \end{itemize}
   \item Organizing and Outlining Your Document
   \begin{itemize}
     \item \small{(group/order)} \normalsize{items,} \small{(organize)} \normalsize{groups, and} \small{(format)} \normalsize{outline.}
   \end{itemize}
   \item Devising a Schedule and a Budget
   \begin{itemize}
     \item estimate the length of each task.
     \item estimate the cost of performing experiments.
   \end{itemize}
\end{itemize}
\vspace{-3mm}


\subsection{Writing}
At this stage of the process, the purpose and organization of your report is already decided. Now you need to craft the words and phrasing for each part of the report. \cite{lumen_course}\\

\subsubsection{Drafting}
get it on paper in an organized manner. The focus is on expanding your “plan” ideas into sentences and paragraphs, not on perfection. \cite{sethperler_video}

% https://www.overleaf.com/learn/latex/Lists
\begin{itemize}
   \item Guidelines
   \begin{itemize}
     \item \normalsize{Start} \small{(with the easiest)\normalsize{,} \small{(draft)}} \normalsize{Quickly, Don’t Stop} \small{(to research)\normalsize{.}} 
   \end{itemize}
   \item Templates
   \begin{itemize}
     \item readers get tired of seeing the same design.
     \item templates can send the wrong message.
   \end{itemize}
   \item Styles
   \begin{itemize}
     \item save time, ensure consistency, and are useful in collaborative writing.
   \end{itemize}
   \item Language COPS
   \begin{itemize}
     \item {\normalsize{C}\small{apitals}\normalsize{,}} 
           {\normalsize{O}\small{verall appearance}\normalsize{,}}
           {\normalsize{P}\small{unctuation}\normalsize{, }\small{and}}
           {\normalsize{S}\small{pelling}\normalsize{.}}
   \end{itemize}
\end{itemize}

\subsubsection{Editing\ \cite{Lec3},\ \cite{Lec4}}
``One of the most important skills a writer can have is the ability to compose clear, complete sentences. The sentence is the basic unit of communication in all forms of English.'' 
\begin{flushright}
    \leavevmode\\[-1.5em]
    \small
    Funk, McMahan, and Day. Elements of Grammar
\end{flushright}

\begin{itemize}
   \item Words
   \item Sentences
   \item Punctuation and Mechanics
\end{itemize}

\begin{flushright}
    \leavevmode\\[-1.5em]
    \small
    (Appendix \ref{App:notes})
\end{flushright}

\subsection{Revising}
Revising is the rearrangement and fine tuning of a fully developed—if not totally completed—draft so that the thesis is aligned with the writer’s purpose. \cite{lumen_course}

% https://www.overleaf.com/learn/latex/Lists
\begin{itemize}
   \item Studying the Document by Yourself \small{(let it set)}
   \begin{itemize}
     \item read it aloud, use checklists, and review a printout.
   \end{itemize}
   \item Using Revision Software
   \begin{itemize}
     \item spell checkers, grammar checkers and thesauri.
   \end{itemize} 
   \item Seeking Help from Someone Else
   \begin{itemize}
     \item consider the edits of subject-matter experts.
   \end{itemize}
\end{itemize}

\section{CVs and Covering Letters\ \cite{Lec5}}
\subsection{Curriculum Vitae}
\subsection{Covering letters}
The cover letter reflects your communication skills and to some extent your personality.
\begin{figure}[!h]
    \centering
    \includegraphics[width=0.8\columnwidth]{Covering-Letter} 
    \caption{Cover Letter Template\cite{CL_Article}}
\end{figure}

\section{Collaborative Writing\ \cite{Lec6}}
\subsection{Advantages and Disadvantages of Collaborative Writing}
\subsection{Patterns of Collaboration}
\subsection{Guidelines for Efficient and Productive Meetings}

\section{Instructions and Manuals\ \cite{Lec7}}
\subsection{Instructions}
\subsection{Manuals}

\section{Instructions and Manuals\ \cite{Lec8}}
\subsection{Preparing an Oral Presentation}
\subsection{Giving the Oral Presentation}
\subsection{Answering Questions after Presentation}
\subsection{Presentation Evaluation}

\appendices
\section{Elite Students' Notes} \label{App:notes}
There are a lot of previous exam tests available on \href{https://drive.google.com/drive/folders/0B8lyRA5rfGgZYXFsLVZSYlJIQWM?usp=sharing}{\underline{CSED Exams}}, For you to benefit the most from them, solutions to previous exams of Dr. Nagia are provided:

\begin{flushright}
    \leavevmode\\[-1.5em]
    \href{https://docs.google.com/document/d/1Y5qmZdFDYHE5ThyHAbgH9ZTh6VpGHEVZWPBRfeGFrjQ/edit?fbclid=IwAR12ECf3oq_gsi1duXE_ja339MjJ5Y7bFQI5lvY3AkGDDsBIWb0rluTXc9o}{\underline{Begawy's attempt}}
\end{flushright}

\section{Elementary Rules\ \cite{Bartleby_Reference}}
Asserting that one must first know the rules to break them, ``Elements of Style'' is a must-have reference book for any student and conscientious writer. Intended for use in which the practice of composition is combined with the study of literature, it gives in brief space the principal requirements of plain English style and concentrates attention on the rules of usage and principles of composition most commonly violated.

\section{Vocabulary}

\begin{table}[H]
\centering
% https://texblog.org/2019/06/03/control-the-width-of-table-columns-tabular-in-latex/
\begin{tabular}{|p{3cm}|p{3cm}|}
 \hline
 &\\[-1em]
 Word & \begin{arabtext} الترجمة \end{arabtext} \\[-1em]
 \hline\hline
 &\\[-1em]
 Accurate & \begin{arabtext} دقيق \end{arabtext} \\[-1em]
 \hline
 &\\[-1em]
 Clear & \begin{arabtext} واضح \end{arabtext} \\[-1em]
 \hline
 &\\[-1em]
 Concise & \begin{arabtext} مختصر \end{arabtext} \\[-1em]
 \hline
 &\\[-1em]
 Coherent & \begin{arabtext} مترابط \end{arabtext} \\[-1em]
 \hline
 &\\[-1em]
 Accessible & \begin{arabtext} سهل الوصول \end{arabtext} \\[-1em]
 \hline
 &\\[-1em]
 Conducted & \begin{arabtext} أجرى ، أدى \end{arabtext} \\[-1em]
 \hline
 &\\[-1em]
 Tensile & \begin{arabtext} شد \end{arabtext} \\[-1em]
 \hline
 &\\[-1em]
 Baseline & \begin{arabtext} حدود \end{arabtext} \\[-1em]
 \hline
 &\\[-1em]
 Ambiguity & \begin{arabtext} غموض ، التباس \end{arabtext} \\[-1em]
 \hline
 &\\[-1em]
 Preliminary & \begin{arabtext} أولي ، تمهيدي \end{arabtext} \\[-1em]
 \hline
 &\\[-1em]
 Attitude & \begin{arabtext} موقف ، سلوك \end{arabtext} \\[-1em]
 \hline
 &\\[-1em]
 Journalistic & \begin{arabtext} صحفي \end{arabtext} \\[-1em]
 \hline
 &\\[-1em]
 Clustering & \begin{arabtext} تجميع \end{arabtext} \\[-1em]
 \hline
 &\\[-1em]
 Branching & \begin{arabtext} تفريع \end{arabtext} \\[-1em]
 \hline
 &\\[-1em]
 Devising & \begin{arabtext} تصميم ، ابتكار \end{arabtext} \\[-1em]
 \hline
 &\\[-1em]
 Tuning & \begin{arabtext} ضبط \end{arabtext} \\[-1em]
 \hline
 &\\[-1em]
 Thesis & \begin{arabtext} أطروحة ، فرضية \end{arabtext} \\[-1em]
 \hline
 &\\[-1em]
 Thesaurus & \begin{arabtext} قاموس \end{arabtext} \\[-1em]
 \hline
 &\\[-1em]
 Others & \begin{arabtext} أخرى \end{arabtext} \\[-1em]
 \hline
 &\\[-1em]
 Others & \begin{arabtext} أخرى \end{arabtext} \\[-1em]
 \hline
 &\\[-1em]
 Others & \begin{arabtext} أخرى \end{arabtext} \\[-1em]
 \hline
 &\\[-1em]
 Others & \begin{arabtext} أخرى \end{arabtext} \\[-1em]
 \hline
 &\\[-1em]
 Others & \begin{arabtext} أخرى \end{arabtext} \\[-1em]
 \hline
 &\\[-1em]
 Others & \begin{arabtext} أخرى \end{arabtext} \\[-1em]
 \hline
 &\\[-1em]
 Others & \begin{arabtext} أخرى \end{arabtext} \\[-1em]
 \hline
 &\\[-1em]
 Others & \begin{arabtext} أخرى \end{arabtext} \\[-1em]
 \hline
 &\\[-1em]
 Others & \begin{arabtext} أخرى \end{arabtext} \\[-1em]
 \hline
 &\\[-1em]
 Conscientious & \begin{arabtext} عنده ضمير \end{arabtext} \\[-1em]
 \hline
 ... & \multicolumn{1}{r}{...} \\
\end{tabular}
\smallskip 
\caption{Translation}
\end{table}     

\begin{thebibliography}{1}

\bibitem{Lec1}``Technical Reports Writing,''\\ Dr. Nagia Ghanem, 2019. [Lecture]: \url{http://bit.ly/Lec1_TW}.
\bibitem{Lec2}``The Writing Process,''\\ Dr. Nagia Ghanem, 2019. [Lecture]: \url{http://bit.ly/Lec2_TW}.
\bibitem{Lec3}``Document Editing I,''\\ Dr. Nagia Ghanem, 2019. [Lecture]: \url{http://bit.ly/Lec3_TW}.
\bibitem{Lec4}``Document Editing II,''\\ Dr. Nagia Ghanem, 2019. [Lecture]: \url{http://bit.ly/Lec4_TW}.
\bibitem{Lec5}``CVs and Covering Letters,''\\ Dr. Nagia Ghanem, 2019. [Lecture]: \url{http://bit.ly/Lec5_TW}.
\bibitem{Lec6}``Collaborative Writing,''\\ Dr. Nagia Ghanem, 2019. [Lecture]: \url{http://bit.ly/Lec6_TW}.
\bibitem{Lec7}``Instructions and Manuals,''\\ Dr. Nagia Ghanem, 2019. [Lecture]: \url{http://bit.ly/Lec7_TW}.
\bibitem{Lec8}``Oral Presentations,''\\ Dr. Nagia Ghanem, 2019. [Lecture]: \url{http://bit.ly/Lec8_TW}.
\\[-1mm]
\bibitem{LecB}``Writing A Good Technical Paper,''\\ Dr. Moustafa Youssef, 2019. [Lecture]: \url{http://bit.ly/LecB_TW}.
\\[-1mm]
\bibitem{reference_springer1}``How to Write Technical Reports,'' Heike Hering, 2019. \\[0mm] [E-book]: \url{https://doi.org/10.1007/978-3-662-58107-0}.
\bibitem{reference_springer2}``The Craft of Scientific Writing,'' Michael Alley, 2018. \\[0mm] [E-book]: \url{https://doi.org/10.1007/978-1-4419-8288-9}.
\bibitem{reference_springer3}``User Guides, Manuals, and Technical Writing,'' Wallwork, 2014. \\[0mm] [E-book]: \url{https://doi.org/10.1007/978-1-4939-0641-3}.
\\[-1mm]
\bibitem{Bartleby_Reference}``The Elements of Style,'' Bartleby, William Strunk, 1918. \\[0mm] [Online]: \url{https://www.bartleby.com/141/}.
\bibitem{Style_Reference}``Effective Technical Writing in the Information Age,'' Joe Schall. \\[0mm] [Online]: \url{https://www.e-education.psu.edu/styleforstudents/node/1787/}.
\\[-1mm]
\bibitem{CV_Article}``Curriculum Vitae Samples,'' Alison Doyle, 2019. \\[0mm] [Article]: \url{http://bit.ly/CV_Sample}.
\bibitem{CL_Article}``Cover Letter Examples,'' Alison Doyle, 2019. \\[0mm] [Article]: \url{http://bit.ly/CL_Example}.
\\[-1mm]
\bibitem{sethperler_video}``How to Use The Writing Process … In plain English!'' Seth Perler. \\[0mm] [Video]: \url{https://sethperler.com/writing-process/}.
\\[-1mm]
\bibitem{lumen_course}``Business Communication Skills for Managers,'' Lumen. \\[0mm] [Online-Course]: \url{http://bit.ly/lumen_course}.
\end{thebibliography}

\end{document}